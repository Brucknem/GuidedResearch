\section{Related Work}
There are many \ITS{} projects emerging, \eg{} Koster \etal{} at the \emph{Testfeld für Autonomes und vernetztes Fahren in Niedersachsen} \cite{koster2017testfeld}. See \eg{} this and that \dots
Also the cooperative perception using infrastructure sensors is an emerging field \cite{arnold2020cooperative}.

None of these provide public information about their camera calibration and stabilization approaches.

Within the \TAADBW{} project one of the other projects provided information about their calibration techniques.
They are too facing the problems of dynamic stabilization and static calibration.
Among their test area near Karlsruhe, Germany they have built a system mostly consisting of cameras.
Multiple optical camera sensors are attached to large poles with overlapping fields ov view \cite{fleck2018towards}.

They too use a high-precision map to calibrate the cameras. 
They differ by assuming the calibration to be static, whereas we present the foundations for an automated approach to self-calibration.
Our approach differs that we are able to recalibrate at runtime, opposed to the a-priori approach they presented.

The approach of using landmarks like lane markings and poles is similar to ours.
They also do a manual selection of landmarks beforehand.

The calibration procedure itself differs as they assume to have the exact world positions of their landmarks.
We optimize for the world positions using the approximation of objects by their center lines. 
A common approach in this field is to minimize the squared distances between objects in the camera stream and their real world objects projections.

A drawback we face currently is that we only optimize the camera pose per pose, opposed to an global optimization strategy. 
This is mostly due to small overlaps in the fields of view \cite{kraemmer2020providentia} between the cameras.
Additionally we face the problem of data association within the multi-view setup. 
Matching pixels over multiple views between the cameras is inherently hard problem \cite{agrawal2008censure},
assigning samples along the center line of the objects to the exact corresponding pixel over multiple views is not feasible.

Additionally solving the camera calibration problem to radar sensors has been solved previously by Schöller \etal{} within our \Providentia{}.